\documentclass[12pt, a4paper]{article}
\usepackage{fullpage}
\usepackage{amsmath}
\usepackage{graphicx}
\usepackage{natbib}
\usepackage[version=3]{mhchem}
\usepackage{indentfirst}
%\usepackage{epstopdf}
%\usepackage{chemstyle}
\title{Add Hg Reactions from PHREEQC Database to PFLOTRAN Database}
\author{Guoping Tang}
\date\today{}

\begin{document}
\maketitle

\begin{abstract}
The Hg related reactions from PHREEQC database (phreeqc-scb.dat) are added into
PFLOTRAN database (hanford.dat). We conduct speciation calculations with the
cases in Dong et al. (2010) with PFLOTRAN and PHREEQC. The calculated results
are in general agreement, indicating the consistency of the two databases and
codes. Minor differences can be introduced when various activity calculation
options are used in PFLOTRAN. For competition for ligands among metals such as
Fe$^{3+}$, Cu$^{2+}$, as shown in Fig. 4 in Dong et al. (2010), differences in
the complexation reactions for these metals between the two databases may
introduce substantial differences in the calculation results.  
\end{abstract}

\section{Reactions}
\subsection{Aqueous Complexation Reactions}
\subsubsection{Hg$^{2+}$ with Inorganic Species}
\begin{align}
%+'HgOH+' 3 1.0 'Hg++' 1.0 'H2O' -1.0 'H+' 500.000 3.3970 500.00 500.000 500.000 500.000 500.000 500.000 4.0 1.0 217.5973 'scb'
& \ce{Hg^2+ + H2O   <->  HgOH+ + H+} &&\mathrm{log\_k} = -3.397 \\
%+'Hg(OH)2' 3 1.0 'Hg++' 2.0 'H2O' -2.0 'H+' 500.000 5.980 500.00 500.000 500.000 500.000 500.000 500.000 4.0 0.0 234.6046 'scb'
& \ce{Hg^2+ + 2 H2O <->  Hg(OH)2 + 2H+}  &&\mathrm{log\_k} = -5.98 \\
%+'Hg(OH)3-' 3 1.0 'Hg++' 3.0 'H2O' -3.0 'H+' 500.000 21.091 500.00 500.000 500.000 500.000 500.000 500.000 4.0 -1.0 251.6119 'scb'
& \ce{Hg^2+ + 3 H2O  <->  Hg(OH)3^- + 3H+} &&\mathrm{log\_k} = -21.091 \\
%+'Hg2OH+++' 3 2.0 'Hg++' 1.0 'H2O' -1.0 'H+' 500.000 3.2970 500.00 500.000 500.000 500.000 500.000 500.000 4.0 3.0 418.1873 'scb'
& \ce{2Hg^2+ + H2O  <->  Hg_2OH^3+ + H+} && \mathrm{log\_k} = -3.297 \\
%+'Hg3(OH)3+++' 3 3.0 'Hg++' 3.0 'H2O' -3.0 'H+' 500.000 6.3910 500.00 500.000 500.000 500.000 500.000 500.000 4.0 3.0 652.7919 'scb'
& \ce{3Hg^2+ + 3H2O  <->  Hg_3(OH)_3^3+ + 3H+} && \mathrm{log\_k} = -6.391 \\
%+'HgBr+' 2 1.0 'Hg++' 1.0 'Br-' 500.000 -9.6100 500.000 500.000 500.000 500.000 500.000 500.000 4.0 1.0 280.494 'scb'
& \ce{Hg^2+ + Br^- <-> HgBr^+} && \mathrm{log\_k} = 9.61 \\
%+'HgBr2' 2 1.0 'Hg++' 2.0 'Br-' 500.000 -18.080 500.000 500.000 500.000 500.000 500.000 500.000 4.0 0.0 360.398 'scb'
& \ce{Hg^2+ + 2Br^- <-> HgBr2} && \mathrm{log\_k} = 18.08 \\
%+'HgBr3-' 2 1.0 'Hg++' 3.0 'Br-' 500.000 -20.510 500.000 500.000 500.000 500.000 500.000 500.000 4.0 -1.0 440.302 'scb'
& \ce{Hg^2+ + 3Br^- <-> HgBr_3^-} && \mathrm{log\_k} = 20.51 \\
%+'HgBr4--' 2 1.0 'Hg++' 4.0 'Br-' 500.000 -21.740 500.000 500.000 500.000 500.000 500.000 500.000 4.0 -2.0 520.206 'scb'
& \ce{Hg^2+ + 4Br^- <-> HgBr_4^2-} && \mathrm{log\_k} = 21.74 \\
%+'HgOHBr' 4 1.0 'Hg++' 1.0 'Br-' 1.0 'H2O' -1.0 'H+' 500.000 -6.2400 500.00 500.000 500.000 500.000 500.000 500.000 4.0 0.0 297.5013 'scb'
& \ce{Hg^2+ + Br^- + H2O <-> HgOHBr + H+} && \mathrm{log\_k} = 6.24 \\
%+'HgBrI' 3 1.0 'Hg++' 1.0 'Br-' 1.0 'I-' 500.000 -21.12 500.00 500.000 500.000 500.000 500.000 500.000 4.0 0.0 407.3985 'scb'
& \ce{Hg^2+ + Br^- + I- <-> HgBrI} && \mathrm{log\_k} = 21.12 \\
%+'HgBrI3--' 3 1.0 'Hg++' 1.0 'Br-' 3.0 'I-' 500.000 -28.020 500.00 500.000 500.000 500.000 500.000 500.000 4.0 -2.0 661.2075 'scb'
& \ce{Hg^2+ + Br^- + 3I- <-> HgBrI_3^2-} && \mathrm{log\_k} = 28.02 \\
%+'HgBr2I2--' 3 1.0 'Hg++' 2.0 'Br-' 2.0 'I-' 500.000 -26.800 500.00 500.000 500.000 500.000 500.000 500.000 4.0 -2.0 614.207 'scb'
& \ce{Hg^2+ + 2Br^- + 2I- <-> HgBr_2I_2^2-} && \mathrm{log\_k} = 26.80 \\
%+'HgCitrate-' 2 1.0 'Hg++' 1.0 'Citrate---' 500.000 -12.1900 500.00 500.000 500.000 500.000 500.000 500.000 4.0 -1.0 389.6913 'scb'
& \ce{Hg^2+ + Citrate^3- <-> HgCitrate-} && \mathrm{log\_k} = 12.19 \\
%+'HgCl+' 2 1.0 'Hg++' 1.0 'Cl-' -6.6100 -7.3100 -5.9000 -5.6500 -5.5000 -5.5000 -5.8000 -6.4000 4.0 1.0 236.0427 'scb+jin'
& \ce{Hg^2+ + Cl^- <-> HgCl^+} && \mathrm{log\_k} = 7.31 \\
%+'HgCl2' 2 1.0 'Hg++' 2.0 'Cl-' -14.160 -14.000 -12.250 -11.410 -10.710 -10.400 -10.300 -10.800 4.0 0.0 271.4954 'scb+jin'
& \ce{Hg^2+ + 2Cl^- <-> HgCl2} && \mathrm{log\_k} = 14.00 \\
%+'HgCl3-' 2 1.0 'Hg++' 3.0 'Cl-' -16.380 -14.925 -14.200 -13.250 -12.460 -12.100 -12.100 -12.700 4.0 -1.0 306.9481 'scb+jin'
& \ce{Hg^2+ + 3Cl^- <-> HgCl_3^-} && \mathrm{log\_k} = 14.925 \\
%+'HgCl4--' 2 1.0 'Hg++' 4.0 'Cl-' 500.000 -15.535 500.000 500.000 500.000 500.000 500.000 500.000 4.0 -2.0 342.4008 'scb'
& \ce{Hg^2+ + 4Cl^- <-> HgCl_4^2-} && \mathrm{log\_k} = 15.535 \\
%+'HgClI' 3 1.0 'Hg++' 1.0 'Cl-' 1.0 'I-' 500.000 -19.158 500.00 500.000 500.000 500.000 500.000 500.000 4.0 0.0 362.9472 'scb'
& \ce{Hg^2+ + Cl^- + I- <-> HgClI} && \mathrm{log\_k} = 19.158 \\
%+'HgOHCl' 4 1.0 'Hg++' 1.0 'Cl-' 1.0 'H2O' -1.0 'H+' 500.000 -4.2500 500.00 500.000 500.000 500.000 500.000 500.000 4.0 0.0 253.05 'scb'
& \ce{Hg^2+ + Cl^- + H2O <-> HgOHCl + H+} && \mathrm{log\_k} = 4.25 \\
%+'HgCO3' 2 1.0 'Hg++' 1.0 'CO3--' 500.000 -11.47 500.00 500.000 500.000 500.000 500.000 500.000 4.0 0.0 260.5992 'scb'
& \ce{Hg^2+ + CO3^2- <-> HgCO3} && \mathrm{log\_k} = 11.47 \\
%+'Hg(CO3)2--' 2 1.0 'Hg++' 2.0 'CO3--' 500.000 -15.042 500.00 500.000 500.000 500.000 500.000 500.000 4.0 -2.0 320.6084 'scb'
& \ce{Hg^2+ + 2 CO3^2- <-> Hg(CO3)_2^2-} && \mathrm{log\_k} = 15.042 \\
%+'HgHCO3+' 3 1.0 'Hg++' 1.0 'CO3--' 1.0 'H+' 500.000 -15.805 500.00 500.000 500.000 500.000 500.000 500.000 4.0 1.0 261.6071 'scb'
& \ce{Hg^2+ + HCO3^- <-> HgHCO3^+} && \mathrm{log\_k} = 15.805 \\
%+'HgOHCO3-' 4 1.0 'Hg++' 1.0 'CO3--' 1.0 'H2O' -1.0 'H+' 500.000 -5.33 500.00 500.000 500.000 500.000 500.000 500.000 4.0 1.0 277.6065 'scb'
& \ce{Hg^2+ + CO_3^2- + H2O <-> HgOHCO3^- + H+} && \mathrm{log\_k} = 5.33 \\
%+'HgEDTA--' 2 1.0 'Hg++' 1.0 'EDTA----' 500.000 -23.2200 500.00 500.000 500.000 500.000 500.000 500.000 4.0 -2.0 488.8034 'scb'
& \ce{Hg^2+ + EDTA^4- <-> HgEDTA^2-} && \mathrm{log\_k} = 23.22 \\
%+'HgHEDTA-' 3 1.0 'Hg++' 1.0 'EDTA----' 1.0 'H+' 500.000 -26.8500 500.00 500.000 500.000 500.000 500.000 500.000 4.0 -1.0 489.8113 'scb'
& \ce{Hg^2+ + EDTA^4- + H+ <-> HgHEDTA^-} && \mathrm{log\_k} = 26.85 \\
%+'HgH2EDTA' 3 1.0 'Hg++' 1.0 'EDTA----' 2.0 'H+' 500.000 -29.1500 500.00 500.000 500.000 500.000 500.000 500.000 4.0 0.0 490.8192 'scb'
& \ce{Hg^2+ + EDTA^4- + 2H+ <-> HgH2EDTA} && \mathrm{log\_k} = 29.15 \\
%+'HgOHEDTA---' 4 1.0 'Hg++' 1.0 'EDTA----' -1.0 'H+' 1.0 'H2O' 500.000 -13.6800 500.00 500.000 500.000 500.000 500.000 500.000 4.0 -3.0 505.8107 'scb'
& \ce{Hg^2+ + EDTA^4- + H2O <-> HgOHEDTA^3- + H+} && \mathrm{log\_k} = 13.68 \\
%+'HgF+' 2 1.0 'Hg++' 1.0 'F-'  500.000 -1.5700 500.000 500.000 500.000 500.000 500.000 500.000 4.0 1.0 219.5884 'scb'
& \ce{Hg^2+ + F^- <-> HgF^+} && \mathrm{log\_k} = 1.57
\end{align}
\begin{align}
%+'HgI+' 2 1.0 'Hg++' 1.0 'I-' 500.000 -13.410 500.000 500.000 500.000 500.000 500.000 500.000 4.0 1.0 327.4945 'scb'
& \ce{Hg^2+ + I^- <-> HgI^+} && \mathrm{log\_k} = 13.41 \\
%+'HgI2' 2 1.0 'Hg++' 2.0 'I-' 500.000 -24.630 500.000 500.000 500.000 500.000 500.000 500.000 4.0 0.0 454.399 'scb'
& \ce{Hg^2+ + 2I^- <-> HgI2} && \mathrm{log\_k} = 24.63 \\
%+'HgI3-' 2 1.0 'Hg++' 3.0 'I-' 500.000 -28.410 500.000 500.000 500.000 500.000 500.000 500.000 4.0 -1.0 581.3035 'scb'
& \ce{Hg^2+ + 3I^- <-> HgI_3^-} && \mathrm{log\_k} = 28.41 \\
%+'HgI4--' 2 1.0 'Hg++' 4.0 'I-' 500.000 -30.340 500.000 500.000 500.000 500.000 500.000 500.000 4.0 -2.0 708.208 'scb'
& \ce{Hg^2+ + 4I^- <-> HgI_4^2-} && \mathrm{log\_k} = 30.34 \\
%+'HgOHI' 4 1.0 'Hg++' 1.0 'I-' 1.0 'H2O' -1.0 'H+' 500.000 -9.4100 500.00 500.000 500.000 500.000 500.000 500.000 4.0 0.0 344.5018 'scb'
& \ce{Hg^2+ + I^- + H2O <-> HgOHI + H+} && \mathrm{log\_k} = 9.41 \\
%+'HgNH3++' 3 1.0 'Hg++' 1.0 'NH4+' -1.0 'H+' 500.000 0.444 500.000 500.000 500.000 500.000 500.000 500.000 4.0 2.0 217.6206 'scb'
& \ce{Hg^2+ + NH4^+ <-> HgNH3^2+ + H+} && \mathrm{log\_k} = -0.444 \\
%+'Hg(NH3)2++' 3 1.0 'Hg++' 2.0 'NH4+' -2.0 'H+' 500.000 0.688 500.000 500.000 500.000 500.000 500.000 500.000 4.0 2.0 234.6512 'scb'
& \ce{Hg^2+ + 2 NH4^+ <-> Hg(NH3)2^2+ + 2H+} && \mathrm{log\_k} = -0.688 \\
%+'Hg(NH3)3++' 3 1.0 'Hg++' 3.0 'NH4+' -3.0 'H+' 500.000 9.432 500.000 500.000 500.000 500.000 500.000 500.000 4.0 2.0 251.6818 'scb'
& \ce{Hg^2+ + 3 NH4^+ <-> Hg(NH3)3^2+ + 3H+} && \mathrm{log\_k} = -9.432 \\
%+'Hg(NH3)4++' 3 1.0 'Hg++' 4.0 'NH4+' -4.0 'H+' 500.000 17.676 500.000 500.000 500.000 500.000 500.000 500.000 4.0 2.0 268.7124 'scb'
& \ce{Hg^2+ + 4 NH4^+ <-> Hg(NH3)4^2+ + 4H+} && \mathrm{log\_k} = -17.676 \\
%+'HgNO3+' 2 1.0 'Hg++' 1.0 'NO3-' 500.000 0.430 500.000 500.000 500.000 500.000 500.000 500.000 4.0 1.0 262.5949 'scb'
& \ce{Hg^2+ + NO_3^- <-> HgNO_3^+} && \mathrm{log\_k} = -0.43 \\
%+'Hg(NO3)2' 2 1.0 'Hg++' 2.0 'NO3-' 500.000 0.810 500.000 500.000 500.000 500.000 500.000 500.000 4.0 0.0 324.5998 'scb'
& \ce{Hg^2+ + 2 NO_3^- <-> Hg(NO_3)_2} && \mathrm{log\_k} = -0.81 \\
%+'HgPO4-' 2 1.0 'Hg++' 1.0 'PO4---' 500.000 -7.841 500.000 500.000 500.000 500.000 500.000 500.000 4.0 -1.0 295.5614 'scb'
& \ce{Hg^2+ + PO_4^3- <-> HgPO_4^-} && \mathrm{log\_k} = 7.841 \\
%+'HgHPO4' 3 1.0 'Hg++' 1.0 'PO4---' 1.0 'H+' 500.000 -20.056 500.000 500.000 500.000 500.000 500.000 500.000 4.0 0.0 296.5693 'scb'
& \ce{Hg^2+ + PO_4^3- + H+ <-> HgHPO_4} && \mathrm{log\_k} = 20.056 \\
%+'HgS' 3 1.0 'Hg++' 1.0 'HS-' -1.0 'H+' 500.000 -28.6 500.00 500.000 500.000 500.000 500.000 500.000 4.0 0.0 232.654 'scb'
& \ce{Hg^2+ + HS^- <-> HgS + H^+} && \mathrm{log\_k} = 28.6 \\
%+'HgS2--' 3 1.0 'Hg++' 2.0 'HS-' -2.0 'H+' 500.000 -24.73 500.00 500.000 500.000 500.000 500.000 500.000 4.0 -2.0 264.718 'scb'
& \ce{Hg^2+ + 2HS^- <-> HgS_2^2- + 2 H^+} && \mathrm{log\_k} = 24.73 \\
%+'Hg(HS)2' 2 1.0 'Hg++' 2.0 'HS-' 500.0 -39.77 500.00 500.000 500.000 500.000 500.000 500.000 4.0 0.0 266.7338 'scb'
& \ce{Hg^2+ + 2HS^- <-> Hg(S_2)_2} && \mathrm{log\_k} = 39.77 \\
%+'HgHS2-' 3 1.0 'Hg++' 2.0 'HS-' -1.0 'H+' 500.0 -33.44 500.00 500.000 500.000 500.000 500.000 500.000 4.0 -1.0 265.7259 'scb'
& \ce{Hg^2+ + 2HS^- <-> HgHS_2^- + H+} && \mathrm{log\_k} = 39.77 \\
%+'HgHS+' 2 1.0 'Hg++' 1.0 'HS-' 500.0 -20.5 500.00 500.000 500.000 500.000 500.000 500.000 4.0 1.0 233.6619 'scb'
& \ce{Hg^2+ + HS^- <-> HgHS^+} && \mathrm{log\_k} = 20.5 \\
%+'HgS5' 5 1.0 'Hg++' 5.0 'HS-' 3.0 'H+' -4.0 'H2O' 2.0 'O2' 500.000 -194.56 500.00 500.000 500.000 500.000 500.000 500.000 4.0 0.0 360.91 'scb'
& \ce{Hg^2+ + 5 HS^- + 3 H^+ 2 O_2 <-> HgS_5 + 4 H2O} && \mathrm{log\_k} = 194.56 \\
%+'Hg(S5)2--' 5 1.0 'Hg++' 10.0 'HS-' 6.0 'H+' -8.0 'H2O' 4.0 'O2' 500.000 -352.0 500.00 500.000 500.000 500.000 500.000 500.000 4.0 -2.0 521.23 'scb'
& \ce{Hg^2+ + 10 HS^- + 6 H^+ 4 O_2 <-> Hg(S_5)_2^2- + 8 H2O} && \mathrm{log\_k} = 352.0 \\
%+'HgS5OH-' 5 1.0 'Hg++' 5.0 'HS-' 2.0 'H+' -3.0 'H2O' 2.0 'O2' 500.000 -185.98 500.00 500.000 500.000 500.000 500.000 500.000 4.0 -1.0 377.9173 'scb'
& \ce{Hg^2+ + 5 HS^- + 2 H^+ 2 O_2 <-> HgS_5OH^- + 3 H2O} && \mathrm{log\_k} = 185.98 \\
%+'Hg(S2O3)2--' 2 1.0 'Hg++' 2.0 'S2O3--' 500.000 -29.48 500.00 500.000 500.000 500.000 500.000 500.000 4.0 -2.0 424.8504 'scb'
& \ce{Hg^2+ + 2 S_2O_3^2- <-> Hg(S2O3)_2^2-} && \mathrm{log\_k} = 29.48 \\
%+'Hg(S2O3)3---' 2 1.0 'Hg++' 3.0 'S2O3--' 500.000 -31.28 500.00 500.000 500.000 500.000 500.000 500.000 4.0 -3.0 536.9806 'scb'
& \ce{Hg^2+ + 3 S_2O_3^2- <-> Hg(S2O3)_3^3-} && \mathrm{log\_k} = 31.28 \\
%+'HgSO4' 2 1.0 'Hg++' 1.0 'SO4--' 500.0 -2.42 500.0 500.0 500.0 500.0 500.0 500.0 4.0 0.0 296.6536 'scb'
& \ce{Hg^2+ + SO_4^2- <-> HgSO4} && \mathrm{log\_k} = 2.42 \\
%+'Hg(SO4)2--' 2 1.0 'Hg++' 2.0 'SO4--' 500.0 -3.48 500.0 500.0 500.0 500.0 500.0 500.0 4.0 -2.0 392.7172 'scb'
& \ce{Hg^2+ + 2 SO_4^2- <-> Hg(SO4)_2^2-} && \mathrm{log\_k} = 3.48
\end{align}

\subsubsection{CH$_4$Hg$^{+}$ with Inorganic Species}
\begin{align}
%+'MmmOH' 3 1.0 'Mmm+' 1.0 'H2O' -1.0 'H+' 500.0 4.557 500.0 500.0 500.0 500.0 500.0 500.0 4.0 0.0 232.6323
& \ce{CH3Hg+ + H2O <-> CH3HgOH + H+} && \mathrm{log\_k} =  -4.557 \\
%+'(Mmm)2OH+' 3 2.0 'Mmm+' 1.0 'H2O' -1.0 'H+' 500.0 1.997 500.0 500.0 500.0 500.0 500.0 500.0 4.0 1.0 448.2573
& \ce{2 CH3Hg+ + H2O <-> (CH3Hg)2OH^+ + H+} && \mathrm{log\_k} =  -1.997 \\
%+'MmmCO3-' 2 1.0 'Mmm+' 1.0 'CO3--' 500.0 -6.511 500.0 500.0 500.0 500.0 500.0 500.0 4.0 -1.0 275.6342
& \ce{CH3Hg+ + CO_3^2- <-> CH3HgCO_3^-} && \mathrm{log\_k} =  6.511 \\
%+'MmmNH3+' 3 1.0 'Mmm+' 1.0 'NH4+' -1.0 'H+' 500.0 1.644 500.0 500.0 500.0 500.0 500.0 500.0 4.0 1.0 232.6556
& \ce{CH3Hg+ + NH_4^+ <-> CH3HgNH_3^+ + H^+} && \mathrm{log\_k} =  -1.644 \\
%+'MmmHPO4-' 3 1.0 'Mmm+' 1.0 'PO4---' 1.0 'H+' 500.0 -17.806 500.0 500.0 500.0 500.0 500.0 500.0 4.0 -1.0 311.6043
& \ce{CH3Hg+ PO_4^3- + H^+ <-> CH3HgHPO_4^-} && \mathrm{log\_k} =  17.806 \\
%+'MmmSO4-' 2 1.0 'Mmm+' 1.0 'SO4--' 500.0 -2.0 500.0 500.0 500.0 500.0 500.0 500.0 4.0 -1.0 311.6886
& \ce{CH3Hg+ + SO_4^2- <-> CH3HgSO_4^-} && \mathrm{log\_k} =  2.0 \\
%+'MmmS-' 3 1.0 'Mmm+' 1.0 'HS-' -1.0 'H+' 500.0 -7.0 500.0 500.0 500.0 500.0 500.0 500.0 4.0 -1.0 247.689
& \ce{CH3Hg+ + HS- <-> CH3HgS^- + H^+} && \mathrm{log\_k} =  7.0 \\
%+'Mmm2S' 3 2.0 'Mmm+' 1.0 'HS-' -1.0 'H+' 500.0 -23.51 500.0 500.0 500.0 500.0 500.0 500.0 4.0 0.0 463.314
& \ce{2 CH3Hg+ + HS- <-> (CH3Hg)_2S + H^+} && \mathrm{log\_k} =  23.51 \\
%+'Mmm3S+' 3 3.0 'Mmm+' 1.0 'HS-' -1.0 'H+' 500.0 -30.51 500.0 500.0 500.0 500.0 500.0 500.0 4.0 1.0 678.939
& \ce{3 CH3Hg+ + HS- <-> (CH3Hg)_3S^+ + H^+} && \mathrm{log\_k} =  30.51 \\
%+'MmmF' 2 1.0 'Mmm+' 1.0 'F-' 500.0 -1.71 500.0 500.0 500.0 500.0 500.0 500.0 4.0 0.0 234.6234
& \ce{CH3Hg+ + F- <-> CH3HgF} && \mathrm{log\_k} =  1.71 \\
%+'MmmCl' 2 1.0 'Mmm+' 1.0 'Cl-' 500.0 -5.39 500.0 500.0 500.0 500.0 500.0 500.0 4.0 0.0 251.0777
& \ce{CH3Hg+ + Cl- <-> CH3HgCl} && \mathrm{log\_k} =  5.39 \\
%+'MmmBr' 2 1.0 'Mmm+' 1.0 'Br-' 500.0 -6.7 500.0 500.0 500.0 500.0 500.0 500.0 4.0 0.0 295.529
& \ce{CH3Hg+ + Br- <-> CH3HgBr} && \mathrm{log\_k} =  6.7 \\
%+'MmmI' 2 1.0 'Mmm+' 1.0 'I-' 500.0 -8.71 500.0 500.0 500.0 500.0 500.0 500.0 4.0 0.0 342.5295
& \ce{CH3Hg+ + I- <-> CH3HgI} && \mathrm{log\_k} =  8.71 \\
%+'MmmI2-' 2 1.0 'Mmm+' 2.0 'I-' 500.0 -9.18 500.0 500.0 500.0 500.0 500.0 500.0 4.0 -1.0 469.434
& \ce{CH3Hg+ + 2 I- <-> CH3HgI_2^-} && \mathrm{log\_k} =  9.18 \\
%+'MmmS2O3-' 2 1.0 'Mmm+' 1.0 'S2O3--' 500.0 -11.18 500.0 500.0 500.0 500.0 500.0 500.0 4.0 -1.0 327.7512
& \ce{CH3Hg+ + S_2O3^2- <-> CH3HgS_2O_3^-} && \mathrm{log\_k} =  11.18
\end{align}

\subsubsection{Hg$^{2+}$ and CH$_4$Hg$^{+}$ with Organic Species}

\begin{align}
%+'HDom---' 2 1.0 'H+' 1.0 'Dom----' 500.0 -10.0 500.0 500.0 500.0 500.0 500.0 500.0 4.0 -3.0 12.0111 'scb'
& \ce{H+ + DOM^4- <-> HDom^3-} && \mathrm{log\_k} = 10.0 \\
%'HgDom--' 2 1.0 'Hg++' 1.0 'Dom----' 500.0 -21.4 500.0 500.0 500.0 500.0 500.0 500.0 4.0 -2.0 212.6011 'scb'
& \ce{Hg^2+ + DOM^4- <-> HgDom^2-} && \mathrm{log\_k} = 21.4 \\
%+'RcooH' 2 1.0 'Rcoo-' 1.0 'H+' 500.0 -4.5 500.0 500.0 500.0 500.0 500.0 500.0 4.0 0.0 13.019 'scb'
& \ce{H+ + Rcoo^- <-> RcooH} && \mathrm{log\_k} = 4.5 \\
%+'HgRcoo+' 2 1.0 'Hg++' 1.0 'Rcoo-' 500.0 -10.0 500.0 500.0 500.0 500.0 500.0 500.0 4.0 1.0 212.6011 'scb'
& \ce{Hg^2+ + Rcoo^- <-> HgRcoo^+} && \mathrm{log\_k} = 10.0 \\
%+'RsH' 2 1.0 'Rs-' 1.0 'H+' 500.0 -10.0 500.0 500.0 500.0 500.0 500.0 500.0 4.0 0.0 33.0719 'scb'
& \ce{H+ + Rs^- <-> RsH} && \mathrm{log\_k} = 10.0 \\
%+'HgRs+' 2 1.0 'Hg++' 1.0 'Rs-' 500.0 -21.0 500.0 500.0 500.0 500.0 500.0 500.0 4.0 1.0 232.654 'scb'
& \ce{Hg^2+ + Rs^- <-> HgRs^+} && \mathrm{log\_k} = 21.0 \\
%+'HgRs2' 2 1.0 'Hg++' 2.0 'Rs-' 500.0 -28.7 500.0 500.0 500.0 500.0 500.0 500.0 4.0 0.0 264.718 'scb'
& \ce{Hg^2+ + 2 Rs^- <-> HgRs_2} && \mathrm{log\_k} = 28.7 \\
%+'MmmRs' 2 1.0 'Mmm+' 1.0 'Rs-' 500.0 -14.0 500.0 500.0 500.0 500.0 500.0 500.0 4.0 0.0 247.689 'scb'
& \ce{CH_3Hg+ + Rs^- <-> CH_3HgRs} && \mathrm{log\_k} = 14.0 \\
%+'HL' 2 1.0 'H+' 1.0 'L-' 500.0 -10.0 500.0 500.0 500.0 500.0 500.0 500.0 4.0 0.0 13.019 'scb'
& \ce{H+ + L^- <-> HL} && \mathrm{log\_k} = 10.0 \\
%+'HgL+' 2 1.0 'Hg++' 1.0 'L-' 500.0 -21.27 500.0 500.0 500.0 500.0 500.0 500.0 4.0 1.0 212.6011 'scb'
& \ce{Hg^2+ + L^- <-> HgL^+} && \mathrm{log\_k} = 21.27 \\
%+'HgL2' 2 1.0 'Hg++' 2.0 'L-' 500.0 -34.04 500.0 500.0 500.0 500.0 500.0 500.0 4.0 0.0 224.6122 'scb'
& \ce{Hg^2+ + 2 L^- <-> HgL_2} && \mathrm{log\_k} = 34.04 \\
%+'MmmL' 2 1.0 'Mmm+' 1.0 'L-' 500.0 -16.76 500.0 500.0 500.0 500.0 500.0 500.0 4.0 0.0 227.6361 'scb'
& \ce{CH_3Hg^+ + L^- <-> CH_3HgL} && \mathrm{log\_k} = 16.76 \\
%+'CaL+' 2 1.0 'Ca++' 1.0 'L-' 500.0   -4.3 500.0 500.0 500.0 500.0 500.0 500.0 4.0 1.0  52.0891 'scb'
& \ce{Ca^2+ + L^- <-> CaL^+} && \mathrm{log\_k} = 4.3 \\
%+'MgL+' 2 1.0 'Mg++' 1.0 'L-' 500.0  -3.61 500.0 500.0 500.0 500.0 500.0 500.0 4.0 1.0  36.3161 'scb'
& \ce{Mg^2+ + L^- <-> MgL^+} && \mathrm{log\_k} = 3.61 \\
%+'ZnL+' 2 1.0 'Zn++' 1.0 'L-' 500.0  -9.69 500.0 500.0 500.0 500.0 500.0 500.0 4.0 1.0  77.4011 'scb'
& \ce{Zn^2+ + L^- <-> ZnL^+} && \mathrm{log\_k} = 9.69 \\
%+'ZnL2' 2 1.0 'Zn++' 2.0 'L-' 500.0 -16.10 500.0 500.0 500.0 500.0 500.0 500.0 4.0 0.0  89.4122 'scb'
& \ce{Zn^2+ + 2 L^- <-> ZnL_2} && \mathrm{log\_k} = 16.10 \\
%+'CuL+' 2 1.0 'Cu++' 1.0 'L-' 500.0  -12.0 500.0 500.0 500.0 500.0 500.0 500.0 4.0 1.0  75.5571 'scb'
& \ce{Cu^2+ + L^- <-> CuL^+} && \mathrm{log\_k} = 12.0 \\
%+'CuL2' 2 1.0 'Cu++' 2.0 'L-' 500.0 -16.95 500.0 500.0 500.0 500.0 500.0 500.0 4.0 0.0  87.5682 'scb'
& \ce{Cu^2+ + 2 L^- <-> CuL_2} && \mathrm{log\_k} = 16.95 \\
%+'NiL+' 2 1.0 'Ni++' 1.0 'L-' 500.0  -9.72 500.0 500.0 500.0 500.0 500.0 500.0 4.0 1.0  70.7011 'scb'
& \ce{Ni^2+ + L^- <-> NiL^+} && \mathrm{log\_k} = 9.72 \\
%+'NiL2' 2 1.0 'Ni++' 2.0 'L-' 500.0 -15.96 500.0 500.0 500.0 500.0 500.0 500.0 4.0 0.0  82.7122 'scb'
& \ce{Ni^2+ + 2 L^- <-> NiL_2} && \mathrm{log\_k} = 15.96 \\
%+'CdL+' 2 1.0 'Cd++' 1.0 'L-' 500.0 -10.62 500.0 500.0 500.0 500.0 500.0 500.0 4.0 1.0 124.4221 'scb'
& \ce{Cd^2+ + L^- <-> CdL^+} && \mathrm{log\_k} = 10.62 \\
%+'CdL2' 2 1.0 'Cd++' 2.0 'L-' 500.0 -17.48 500.0 500.0 500.0 500.0 500.0 500.0 4.0 0.0 136.4332 'scb'
& \ce{Cd^2+ + 2 L^- <-> CdL_2} && \mathrm{log\_k} = 17.48 \\
%+'PbL+' 2 1.0 'Pb++' 1.0 'L-' 500.0 -11.83 500.0 500.0 500.0 500.0 500.0 500.0 4.0 1.0 219.2111 'scb'
& \ce{Pb^2+ + L^- <-> PbL^+} && \mathrm{log\_k} = 11.83 \\
%+'PbL2' 2 1.0 'Pb++' 2.0 'L-' 500.0 -15.86 500.0 500.0 500.0 500.0 500.0 500.0 4.0 0.0 231.2222 'scb'
& \ce{Pb^2+ + 2 L^- <-> PbL_2} && \mathrm{log\_k} = 15.86 \\
%+'CoL+' 2 1.0 'Co++' 1.0 'L-' 500.0 -10.62 500.0 500.0 500.0 500.0 500.0 500.0 4.0 1.0  70.9443 'scb'
& \ce{Co^2+ + L^- <-> CoL^+} && \mathrm{log\_k} = 10.62 \\
%+'CoL2' 2 1.0 'Co++' 2.0 'L-' 500.0 -17.48 500.0 500.0 500.0 500.0 500.0 500.0 4.0 0.0  82.9554 'scb'
& \ce{Co^2+ + 2 L^- <-> CoL_2} && \mathrm{log\_k} = 17.48 \\
%+'UO2L2' 2 1.0 'UO2++' 2.0 'L-' 500.0 -10.19 500.0 500.0 500.0 500.0 500.0 500.0 4.0 0.0 294.0499 'scb'
& \ce{UO_2^2+ + 2 L^- <-> UO_2L_2} && \mathrm{log\_k} = 10.19 \\
%+'FeL++' 2 1.0 'Fe+++' 1.0 'L-' 500.0 -12.28 500.0 500.0 500.0 500.0 500.0 500.0 4.0 2.0  67.8581 'scb'
& \ce{Fe^3+ + L^- <-> FeL^2+} && \mathrm{log\_k} = 12.28 \\
%+'FeL2+' 2 1.0 'Fe+++' 2.0 'L-' 500.0 -16.40 500.0 500.0 500.0 500.0 500.0 500.0 4.0 1.0  79.8692 'scb'
& \ce{Fe^3+ + 2 L^- <-> FeL_2^+} && \mathrm{log\_k} = 16.40
\end{align}

\subsubsection{Other Metals}
%+'Cu(OH)2' 3 -2.0000 'H+' 1.0000 'Cu++' 2.0000 'H2O' 500.0000 13.68 500.0000 500.0000 500.0000 500.0000 500.0000 500.0000 4.0 1.0 97.5606 'scb' 
\subsection{Surface Complexation Reactions}
\subsection{Precipitation and Dissolution  Reactions}

\section{Example Calculations}

\subsection{Hg$^{2+}$ Speciation with Inorganic Ligands in EFPC Waters}
Species and concentrations from Dong et al. (2010) are used in the calculations. 
\begin{figure}[ht]
\centering
\includegraphics[width=0.6\textwidth]{../pflotran/speciation/Dong2010/ex1/comp.pdf}
\caption{
Hg$^{2+}$ speciation with inorganic ligands in EFPC waters. Total
Hg$^{2+}$ = 0.4 nM. Other inorganic species and average concentrations from Dong
et al. (2010) are used in the calculations. Curves = PHREEQC; Unfilled symbols
= PFLOTRAN default (unit activity coefficient). PFLOTRAN and PHREEQC
calculations are in general agreement.
}
\label{fig1}
\end{figure}

\newpage
\subsection{Hg$^{2+}$ Speciation with Organic Ligands in EFPC Waters}
\begin{figure}[ht]
\centering
\includegraphics[width=0.6\textwidth]{../pflotran/speciation/Dong2010/ex2/ex2.pdf}
\caption{
Hg$^{2+}$ Speciation with organic ligands in EFPC waters. Based on
inorganic species and concentrations in Fig. \ref{fig1}, 4 nM thiol sites (Rs$^-$) and
16 $\mu$M carboxyl sites (Rcoo$^-$) are added. Curves = PHREEQC; Unfilled
symbols = PFLOTRAN default (unit activity coefficient); Filled symbols = PFLOTRAN +
Debye-Hückel activity coefficient updated in each Newton Iteration. 
The results are quite different from Fig. 1 in Dong et al. (2010) probably
because of different ways in fixing pH. In our calculations, HCl or NaOH is
assumed to be added to a specified pH.
}
\label{fig2}
\end{figure}

\newpage
\subsection{CH$_3$Hg Speciation in EFPC Waters}
\begin{figure}[ht]
\centering
\includegraphics[width=0.6\textwidth]{../pflotran/speciation/Dong2010/ex3/ex3.pdf}
\caption{
CH$_3$Hg speciation in EFPC waters. 5 pM CH$_3$Hg is added into the
solutions as in Fig. \ref{fig2}. The results are in general agreement with Dong
et al. (2010) Fig. 3a except that CH$_3$Hg concentrations are higher in the
latter.}
\label{fig3}
\end{figure}

\newpage
\subsection{Competition with Other Metal Ions for Complexation with Thiol Ligands}
\begin{figure}[ht]
\centering
\includegraphics[width=0.6\textwidth]{../pflotran/speciation/Dong2010/ex4/ex40.pdf}
\caption{
Without competition with other metal ions for complexation with thiol
ligands (L), PFLOTRAN calculations are close to PHREEQC.
The calculations are based on the same input files for Fig. \ref{fig2} except
that the 4 nM thiol sites (Rs$^-$) and 16 $\mu$M carboxyl sites (Rcoo$^-$) are
replaced with 4 nM thiol sites represented by L (thiol group in Cysteine and
Glutathione). For PHREEQC, CH$_3$Hg$^+$, Ni$^{2+}$, Cd$^{2+}$, Pb$^{2+}$,
UO${_2}^{2+}$, and Fe$^{3+}$ are omitted from the solution to prevent the
competition. For PFLOTRAN, the secondary species between these metals and L are
not added in the input file to avoid the competition. 
}
\label{fig40}
\end{figure}

\newpage
\begin{figure}[ht]
\centering
\includegraphics[width=0.6\textwidth]{../pflotran/speciation/Dong2010/ex4/ex4.pdf}
\caption{
With competition with other metal ions for complexation with thiol ligands (L):
PFLOTRAN calculations (add metal thiol complexation reactions to Fig. \ref{fig40} case, empty markers)
are different from PHREEQC (add metal species, curves). Adding secondary species Fe(OH)$_3$(aq),
Fe(OH)$^{2+}$, Fe(OH)${_4}{^-}$, Fe(OH)$^{2+}$, FeF${_2}{^+}$, and Cu(OH)$_2$ into
PFLOTRAN input files gets the PFLOTRAN calculations (filled markers) closer to
PHREEQC. Namely, without complexation reactions of Fe$^{3+}$ and Cu$^{2+}$ with
OH$^-$, more L$^-$ forms complex with Fe$^{3+}$ and Cu$^{2+}$, and less L$^-$
is available to form complex with Hg$^{2+}$.
}
\label{fig4}
\end{figure}


%\newpage
\clearpage
%\cleardoublepage
%\bibliographystyle{plain}
%\bibliographystyle{plainnat}
%\bibliography{monod}

\end{document}
